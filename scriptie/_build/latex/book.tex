%% Generated by Sphinx.
\def\sphinxdocclass{jupyterBook}
\documentclass[letterpaper,10pt,english]{jupyterBook}
\ifdefined\pdfpxdimen
   \let\sphinxpxdimen\pdfpxdimen\else\newdimen\sphinxpxdimen
\fi \sphinxpxdimen=.75bp\relax
%% turn off hyperref patch of \index as sphinx.xdy xindy module takes care of
%% suitable \hyperpage mark-up, working around hyperref-xindy incompatibility
\PassOptionsToPackage{hyperindex=false}{hyperref}
%% memoir class requires extra handling
\makeatletter\@ifclassloaded{memoir}
{\ifdefined\memhyperindexfalse\memhyperindexfalse\fi}{}\makeatother

\PassOptionsToPackage{warn}{textcomp}

\catcode`^^^^00a0\active\protected\def^^^^00a0{\leavevmode\nobreak\ }
\usepackage{cmap}
\usepackage{fontspec}
\defaultfontfeatures[\rmfamily,\sffamily,\ttfamily]{}
\usepackage{amsmath,amssymb,amstext}
\usepackage{polyglossia}
\setmainlanguage{english}



\setmainfont{FreeSerif}[
  Extension      = .otf,
  UprightFont    = *,
  ItalicFont     = *Italic,
  BoldFont       = *Bold,
  BoldItalicFont = *BoldItalic
]
\setsansfont{FreeSans}[
  Extension      = .otf,
  UprightFont    = *,
  ItalicFont     = *Oblique,
  BoldFont       = *Bold,
  BoldItalicFont = *BoldOblique,
]
\setmonofont{FreeMono}[
  Extension      = .otf,
  UprightFont    = *,
  ItalicFont     = *Oblique,
  BoldFont       = *Bold,
  BoldItalicFont = *BoldOblique,
]


\usepackage[Bjarne]{fncychap}
\usepackage[,numfigreset=0,mathnumfig]{sphinx}

\fvset{fontsize=\small}
\usepackage{geometry}


% Include hyperref last.
\usepackage{hyperref}
% Fix anchor placement for figures with captions.
\usepackage{hypcap}% it must be loaded after hyperref.
% Set up styles of URL: it should be placed after hyperref.
\urlstyle{same}

\addto\captionsenglish{\renewcommand{\contentsname}{Voorblad}}

\usepackage{sphinxmessages}



         \usepackage[Latin,Greek]{ucharclasses}
        \usepackage{unicode-math}
        % fixing title of the toc
        \addto\captionsenglish{\renewcommand{\contentsname}{Contents}}
        

\title{Verkeersongelukken}
\date{May 13, 2021}
\release{}
\author{Stijn Melchior Willem Hering}
\newcommand{\sphinxlogo}{\vbox{}}
\renewcommand{\releasename}{}
\makeindex
\begin{document}

\pagestyle{empty}
\sphinxmaketitle
\pagestyle{plain}
\sphinxtableofcontents
\pagestyle{normal}
\phantomsection\label{\detokenize{voorpagina::doc}}


\begin{sphinxVerbatim}[commandchars=\\\{\}]
\PYG{k+kn}{import} \PYG{n+nn}{os}
\PYG{k+kn}{import} \PYG{n+nn}{pandas} \PYG{k}{as} \PYG{n+nn}{pd}
\PYG{k+kn}{import} \PYG{n+nn}{numpy} \PYG{k}{as} \PYG{n+nn}{np}
\PYG{k+kn}{import} \PYG{n+nn}{plotly}\PYG{n+nn}{.}\PYG{n+nn}{express} \PYG{k}{as} \PYG{n+nn}{px}

\PYG{n}{os}\PYG{o}{.}\PYG{n}{getcwd}\PYG{p}{(}\PYG{p}{)}

\PYG{c+c1}{\PYGZsh{} !wget  https://www.rijkswaterstaat.nl/apps/geoservices/geodata/dmc/bron/01\PYGZhy{}01\PYGZhy{}2010\PYGZus{}31\PYGZhy{}12\PYGZhy{}2019.zip  }
\end{sphinxVerbatim}

\begin{sphinxVerbatim}[commandchars=\\\{\}]
\PYGZsq{}/mnt/c/Users/stijn/OneDrive/Bureaublad/Afstudeerproject/afstudeerproject\PYGZus{}repo/scriptie\PYGZsq{}
\end{sphinxVerbatim}

\begin{sphinxShadowBox}
\sphinxstylesidebartitle{}

\begin{sphinxadmonition}{note}{Note:}
\sphinxAtStartPar
Here is a note!
\end{sphinxadmonition}
\end{sphinxShadowBox}

\begin{sphinxVerbatim}[commandchars=\\\{\}]
\PYG{c+c1}{\PYGZsh{} puntlocatie = pd.read\PYGZus{}csv(\PYGZsq{}/Afstudeerproject/Data/01\PYGZhy{}01\PYGZhy{}2010\PYGZus{}31\PYGZhy{}12\PYGZhy{}2019 (BRON)/PGS0112\PYGZhy{}o\PYGZhy{}CSV\PYGZhy{}bestand\PYGZhy{}J\PYGZhy{}1\PYGZhy{}N\PYGZhy{}J\PYGZhy{}N/Netwerkgegevens/puntlocaties.txt\PYGZsq{})}
\end{sphinxVerbatim}

\begin{sphinxVerbatim}[commandchars=\\\{\}]
\PYG{c+c1}{\PYGZsh{} puntlocatie}
\end{sphinxVerbatim}

\begin{sphinxVerbatim}[commandchars=\\\{\}]
\PYG{c+c1}{\PYGZsh{} \PYGZsh{} usecols=[\PYGZsq{}WVK\PYGZus{}BEGDAT\PYGZsq{}, \PYGZsq{}FK\PYGZus{}VELD5\PYGZsq{}, \PYGZsq{}KLOK\PYGZus{}BEG\PYGZsq{}, \PYGZsq{}KLOK\PYGZus{}END\PYGZsq{}]}
\PYG{c+c1}{\PYGZsh{} wegvakken = pd.read\PYGZus{}csv(\PYGZsq{}Verkeersongevallen/01\PYGZhy{}01\PYGZhy{}2010\PYGZus{}31\PYGZhy{}12\PYGZhy{}2019/PGS0112\PYGZhy{}o\PYGZhy{}CSV\PYGZhy{}bestand\PYGZhy{}J\PYGZhy{}1\PYGZhy{}N\PYGZhy{}J\PYGZhy{}N/Netwerkgegevens/wegvakken.txt\PYGZsq{})}

\PYG{c+c1}{\PYGZsh{} wegvakken.columns}
\end{sphinxVerbatim}

\begin{sphinxVerbatim}[commandchars=\\\{\}]
\PYG{c+c1}{\PYGZsh{} ongevallen = pd.read\PYGZus{}csv(\PYGZsq{}Verkeersongevallen/01\PYGZhy{}01\PYGZhy{}2010\PYGZus{}31\PYGZhy{}12\PYGZhy{}2019/PGS0112\PYGZhy{}o\PYGZhy{}CSV\PYGZhy{}bestand\PYGZhy{}J\PYGZhy{}1\PYGZhy{}N\PYGZhy{}J\PYGZhy{}N/Ongevallengegevens/ongevallen.txt\PYGZsq{})}
\PYG{c+c1}{\PYGZsh{} display(ongevallen.shape)}
\PYG{c+c1}{\PYGZsh{} display(ongevallen.columns)}
\PYG{c+c1}{\PYGZsh{} ongevallen.head()}
\end{sphinxVerbatim}


\part{Voorblad}


\chapter{Abstract}
\label{\detokenize{abstract:abstract}}\label{\detokenize{abstract::doc}}
\sphinxAtStartPar
hier komt de abstract


\chapter{Inhoudsopgave}
\label{\detokenize{inhoudsopgave:inhoudsopgave}}\label{\detokenize{inhoudsopgave::doc}}
\sphinxAtStartPar
There are many ways to write content in Jupyter Book. This short section
covers a few tips for how to do so.


\part{Introductie}


\chapter{Introductie}
\label{\detokenize{introductie:introductie}}\label{\detokenize{introductie::doc}}
\sphinxAtStartPar
Ondanks de ‘Covid\sphinxhyphen{}19\sphinxhyphen{}lockdown’ en de restricties die daar bij horen wordt de doelstelling van 2020 omtrent ernstige verkeersgewonden en doden niet gehaald. Dit blijkt uit de voorlopige cijfers van de Stichting Wetenschappelijk Onderzoek Verkeersveiligheid (SWOV, 2020). Ondanks de verplichte verminderde mobiliteit en het noodzakelijke thuiswerken in 2020, zullen er volgens het Centraal Bureau voor de Statistiek (CBS, 2020a) naar schatting toch méér verkeersdoden geregistreerd worden dan de doelstelling van maximaal 500.
Sinds 1999 is er een dalende trend zichtbaar in het aantal doden. Van 1999 tot 2010 halveerde het aantal verkeersdoden. De afgelopen tien jaar stagneerde dit cijfer. Het bleef gemiddeld rond de 600 verkeersdoden. Wanneer we deze cijfers vergelijken met het aantal ernstige verkeersgewonden is het beeld nóg minder positief. Met 19.700 ernstige verkeersgewonden in 2011 en 21.400 in 2019 (Rijkswaterstaat, 2020) is er in de afgelopen tien jaar een positieve trend. Dit is het bijna het dubbele dan de vastgestelde doelstelling; maximaal 10.600 ernstig verkeersgewonden in 2020. Uit de voorlopige cijfers van het Bestand Geregistreerde Ongevallen in Nederland (BRON) en de Stichting Wetenschappelijk Onderzoek Verkeersveiligheid (SWOV, 2020) is de verwachting van de Rijkswaterstaat dat deze doelstelling in 2020 ook niet worden gehaald (Rijkswaterstaat, 2020).
Volgens het verkeersveiligheidsrapport 2020 van de SWOV bestaat de grootste groep verkeersdoden en verkeersgewonden uit enkelvoudige ongevallen(ongeval zonder betrokkenheid van een andere verkeersdeelnemer) als gevolg van een aanrijding met auto (SWOV, 2020). Daarnaast is er ook een significante stijging te zien in het aantal enkelvoudige ongevallen, deze worden met name veroorzaakt door het aantal fietsongevallen.
Het verminderen van het aantal verkeersdoden en gewonden is van belang als Nederland zich aan de nationale ambities wilt houden; de halvering van doden en ernstig verkeersgewonden in 2030 (SVOW, 2020). Om dit doel te behalen is het essentieel dat er wordt geëxperimenteerd met meerdere innovatieve ideeën om het aantal verkeersincidenten te verminderen. Volgens de SWOV (2020) zijn er ondanks de maatregelen uit het regeerakkoord, de investeringsimpuls en de risico gestuurde aanpak vanuit het Strategisch Plan Verkeersveiligheid 2030 meer en effectievere maatregelen nodig. Dit onderzoek aangaande het analyseren\sphinxhyphen{}en voorspellen van verkeersongelukken aan de hand van het BRON, moet helpen om de gemaakt doelstelling voor 2030 te bereiken en daarmee de verkeersveiligheid in Nederland te verbeteren.


\section{Tussenstuk test}
\label{\detokenize{introductie:tussenstuk-test}}
\sphinxAtStartPar
De Nederlandse economie en samenleving leunt zwaar op transport en logistiek. Dit blijkt uit cijfers van het CBS (2020b).  Veilige wegen zijn hierdoor van groots belang. Het doel van dit onderzoek is om kennis en inzicht te krijgen over de factoren die invloed hebben op verkeersincidenten. Door middel van kwantitatief onderzoek, via data\sphinxhyphen{}analyse en het maken van een voorspellingsmodel, moet deze kennis worden verkregen. Uit de data\sphinxhyphen{}analyse zal duidelijk worden welke factoren en omstandigheden het meeste impact hebben op de kans op verkeersongelukken. Deze factoren kunnen worden gebruikt in een voorspellend model. Aan de hand van dit model moet duidelijk worden waar, wanneer en met welke factoren de kans op verkeersongelukken het grootst is. Met deze kennis kunnen Nederlandse wegen veiliger gemaakt worden of kunnen huldiensten eerder op locatie aanwezig zijn omdat ze strategisch beter gestationeerd worden.
De SWOV beschikt over een grote hoeveelheid kennis over het thema verkeersveiligheid. Deze kennis is voor een groot deel openbaar beschikbaar via de website van het SWOV.  Alleen de nodige relevante kennis over ‘predictive modelling’ in het verkeer ontbreekt. Een groot aandeel van de artikelen die corresponderen met dit onderwerp en door de SWOV zijn gepubliceerd, dateren van 1990  tot 2010. Meer recentere publicaties gaan voornamelijk over ‘predictive modelling’ van de tijd tussen een verkeersongeluk en het moment dat de (snel)weg weer vrij is en het modeleren van automatische verkeersongeluk detectie. Dit onderzoek moet bijdragen aan een meer actuele kennis over ‘predictive modelling’ op het gebied van verkeersongelukken, waardoor de algemene kennis omtrent verkeersveiligheid die via de SWOV momenteel beschikbaar is kan worden uitgebreid.


\section{Markdown Files}
\label{\detokenize{markdown:markdown-files}}\label{\detokenize{markdown::doc}}
\sphinxAtStartPar
Whether you write your book’s content in Jupyter Notebooks (\sphinxcode{\sphinxupquote{.ipynb}}) or
in regular markdown files (\sphinxcode{\sphinxupquote{.md}}), you’ll write in the same flavor of markdown
called \sphinxstylestrong{MyST Markdown}.


\subsection{What is MyST?}
\label{\detokenize{markdown:what-is-myst}}
\sphinxAtStartPar
MyST stands for “Markedly Structured Text”. It
is a slight variation on a flavor of markdown called “CommonMark” markdown,
with small syntax extensions to allow you to write \sphinxstylestrong{roles} and \sphinxstylestrong{directives}
in the Sphinx ecosystem.


\subsection{What are roles and directives?}
\label{\detokenize{markdown:what-are-roles-and-directives}}
\sphinxAtStartPar
Roles and directives are two of the most powerful tools in Jupyter Book. They
are kind of like functions, but written in a markup language. They both
serve a similar purpose, but \sphinxstylestrong{roles are written in one line}, whereas
\sphinxstylestrong{directives span many lines}. They both accept different kinds of inputs,
and what they do with those inputs depends on the specific role or directive
that is being called.


\subsubsection{Using a directive}
\label{\detokenize{markdown:using-a-directive}}
\sphinxAtStartPar
At its simplest, you can insert a directive into your book’s content like so:

\begin{sphinxVerbatim}[commandchars=\\\{\}]
```\PYGZob{}mydirectivename\PYGZcb{}
My directive content
```
\end{sphinxVerbatim}

\sphinxAtStartPar
This will only work if a directive with name \sphinxcode{\sphinxupquote{mydirectivename}} already exists
(which it doesn’t). There are many pre\sphinxhyphen{}defined directives associated with
Jupyter Book. For example, to insert a note box into your content, you can
use the following directive:

\begin{sphinxVerbatim}[commandchars=\\\{\}]
```\PYGZob{}note\PYGZcb{}
Here is a note
```
\end{sphinxVerbatim}

\sphinxAtStartPar
This results in:

\begin{sphinxadmonition}{note}{Note:}
\sphinxAtStartPar
Here is a note
\end{sphinxadmonition}

\sphinxAtStartPar
In your built book.

\sphinxAtStartPar
For more information on writing directives, see the
\sphinxhref{https://myst-parser.readthedocs.io/}{MyST documentation}.


\subsubsection{Using a role}
\label{\detokenize{markdown:using-a-role}}
\sphinxAtStartPar
Roles are very similar to directives, but they are less\sphinxhyphen{}complex and written
entirely on one line. You can insert a role into your book’s content with
this pattern:

\begin{sphinxVerbatim}[commandchars=\\\{\}]
Some content \PYGZob{}rolename\PYGZcb{}`and here is my role\PYGZsq{}s content!`
\end{sphinxVerbatim}

\sphinxAtStartPar
Again, roles will only work if \sphinxcode{\sphinxupquote{rolename}} is a valid role’s name. For example,
the \sphinxcode{\sphinxupquote{doc}} role can be used to refer to another page in your book. You can
refer directly to another page by its relative path. For example, the
role syntax \sphinxcode{\sphinxupquote{\{doc\}`introductie`}} will result in: {\hyperref[\detokenize{introductie::doc}]{\sphinxcrossref{\DUrole{doc}{Introductie}}}}.

\sphinxAtStartPar
For more information on writing roles, see the
\sphinxhref{https://myst-parser.readthedocs.io/}{MyST documentation}.


\subsubsection{Adding a citation}
\label{\detokenize{markdown:adding-a-citation}}
\sphinxAtStartPar
You can also cite references that are stored in a \sphinxcode{\sphinxupquote{bibtex}} file. For example,
the following syntax: \sphinxcode{\sphinxupquote{\{cite\}`holdgraf\_evidence\_2014`}} will render like
this: \sphinxcite{markdown:id3}.

\sphinxAtStartPar
Moreoever, you can insert a bibliography into your page with this syntax:
The \sphinxcode{\sphinxupquote{\{bibliography\}}} directive must be used for all the \sphinxcode{\sphinxupquote{\{cite\}}} roles to
render properly.
For example, if the references for your book are stored in \sphinxcode{\sphinxupquote{references.bib}},
then the bibliography is inserted with:

\begin{sphinxVerbatim}[commandchars=\\\{\}]
```\PYGZob{}bibliography\PYGZcb{}
```
\end{sphinxVerbatim}

\sphinxAtStartPar
Resulting in a rendered bibliography that looks like:

\sphinxAtStartPar



\subsubsection{Executing code in your markdown files}
\label{\detokenize{markdown:executing-code-in-your-markdown-files}}
\sphinxAtStartPar
If you’d like to include computational content inside these markdown files,
you can use MyST Markdown to define cells that will be executed when your
book is built. Jupyter Book uses \sphinxstyleemphasis{jupytext} to do this.

\sphinxAtStartPar
First, add Jupytext metadata to the file. For example, to add Jupytext metadata
to this markdown page, run this command:

\begin{sphinxVerbatim}[commandchars=\\\{\}]
\PYG{n}{jupyter}\PYG{o}{\PYGZhy{}}\PYG{n}{book} \PYG{n}{myst} \PYG{n}{init} \PYG{n}{markdown}\PYG{o}{.}\PYG{n}{md}
\end{sphinxVerbatim}

\sphinxAtStartPar
Once a markdown file has Jupytext metadata in it, you can add the following
directive to run the code at build time:

\begin{sphinxVerbatim}[commandchars=\\\{\}]
```\PYGZob{}code\PYGZhy{}cell\PYGZcb{}
print(\PYGZdq{}Here is some code to execute\PYGZdq{})
```
\end{sphinxVerbatim}

\sphinxAtStartPar
When your book is built, the contents of any \sphinxcode{\sphinxupquote{\{code\sphinxhyphen{}cell\}}} blocks will be
executed with your default Jupyter kernel, and their outputs will be displayed
in\sphinxhyphen{}line with the rest of your content.

\sphinxAtStartPar
For more information about executing computational content with Jupyter Book,
see \sphinxhref{https://myst-nb.readthedocs.io/}{The MyST\sphinxhyphen{}NB documentation}.


\section{Content with notebooks}
\label{\detokenize{notebooks:content-with-notebooks}}\label{\detokenize{notebooks::doc}}
\sphinxAtStartPar
You can also create content with Jupyter Notebooks. This means that you can include
code blocks and their outputs in your book.


\subsection{Markdown + notebooks}
\label{\detokenize{notebooks:markdown-notebooks}}
\sphinxAtStartPar
As it is markdown, you can embed images, HTML, etc into your posts!

\sphinxAtStartPar


\sphinxAtStartPar
You an also \(add_{math}\) and
\begin{equation*}
\begin{split}
math^{blocks}
\end{split}
\end{equation*}
\sphinxAtStartPar
or
\begin{equation*}
\begin{split}
\begin{aligned}
\mbox{mean} la_{tex} \\ \\
math blocks
\end{aligned}
\end{split}
\end{equation*}
\sphinxAtStartPar
But make sure you \$Escape \$your \$dollar signs \$you want to keep!


\subsection{MyST markdown}
\label{\detokenize{notebooks:myst-markdown}}
\sphinxAtStartPar
MyST markdown works in Jupyter Notebooks as well. For more information about MyST markdown, check
out \sphinxhref{https://jupyterbook.org/content/myst.html}{the MyST guide in Jupyter Book},
or see \sphinxhref{https://myst-parser.readthedocs.io/en/latest/}{the MyST markdown documentation}.


\subsection{Code blocks and outputs}
\label{\detokenize{notebooks:code-blocks-and-outputs}}
\sphinxAtStartPar
Jupyter Book will also embed your code blocks and output in your book.
For example, here’s some sample Matplotlib code:

\begin{sphinxVerbatim}[commandchars=\\\{\}]
\PYG{k+kn}{from} \PYG{n+nn}{matplotlib} \PYG{k+kn}{import} \PYG{n}{rcParams}\PYG{p}{,} \PYG{n}{cycler}
\PYG{k+kn}{import} \PYG{n+nn}{matplotlib}\PYG{n+nn}{.}\PYG{n+nn}{pyplot} \PYG{k}{as} \PYG{n+nn}{plt}
\PYG{k+kn}{import} \PYG{n+nn}{numpy} \PYG{k}{as} \PYG{n+nn}{np}
\PYG{n}{plt}\PYG{o}{.}\PYG{n}{ion}\PYG{p}{(}\PYG{p}{)}
\end{sphinxVerbatim}

\begin{sphinxVerbatim}[commandchars=\\\{\}]
\PYG{c+c1}{\PYGZsh{} Fixing random state for reproducibility}
\PYG{n}{np}\PYG{o}{.}\PYG{n}{random}\PYG{o}{.}\PYG{n}{seed}\PYG{p}{(}\PYG{l+m+mi}{19680801}\PYG{p}{)}

\PYG{n}{N} \PYG{o}{=} \PYG{l+m+mi}{10}
\PYG{n}{data} \PYG{o}{=} \PYG{p}{[}\PYG{n}{np}\PYG{o}{.}\PYG{n}{logspace}\PYG{p}{(}\PYG{l+m+mi}{0}\PYG{p}{,} \PYG{l+m+mi}{1}\PYG{p}{,} \PYG{l+m+mi}{100}\PYG{p}{)} \PYG{o}{+} \PYG{n}{np}\PYG{o}{.}\PYG{n}{random}\PYG{o}{.}\PYG{n}{randn}\PYG{p}{(}\PYG{l+m+mi}{100}\PYG{p}{)} \PYG{o}{+} \PYG{n}{ii} \PYG{k}{for} \PYG{n}{ii} \PYG{o+ow}{in} \PYG{n+nb}{range}\PYG{p}{(}\PYG{n}{N}\PYG{p}{)}\PYG{p}{]}
\PYG{n}{data} \PYG{o}{=} \PYG{n}{np}\PYG{o}{.}\PYG{n}{array}\PYG{p}{(}\PYG{n}{data}\PYG{p}{)}\PYG{o}{.}\PYG{n}{T}
\PYG{n}{cmap} \PYG{o}{=} \PYG{n}{plt}\PYG{o}{.}\PYG{n}{cm}\PYG{o}{.}\PYG{n}{coolwarm}
\PYG{n}{rcParams}\PYG{p}{[}\PYG{l+s+s1}{\PYGZsq{}}\PYG{l+s+s1}{axes.prop\PYGZus{}cycle}\PYG{l+s+s1}{\PYGZsq{}}\PYG{p}{]} \PYG{o}{=} \PYG{n}{cycler}\PYG{p}{(}\PYG{n}{color}\PYG{o}{=}\PYG{n}{cmap}\PYG{p}{(}\PYG{n}{np}\PYG{o}{.}\PYG{n}{linspace}\PYG{p}{(}\PYG{l+m+mi}{0}\PYG{p}{,} \PYG{l+m+mi}{1}\PYG{p}{,} \PYG{n}{N}\PYG{p}{)}\PYG{p}{)}\PYG{p}{)}


\PYG{k+kn}{from} \PYG{n+nn}{matplotlib}\PYG{n+nn}{.}\PYG{n+nn}{lines} \PYG{k+kn}{import} \PYG{n}{Line2D}
\PYG{n}{custom\PYGZus{}lines} \PYG{o}{=} \PYG{p}{[}\PYG{n}{Line2D}\PYG{p}{(}\PYG{p}{[}\PYG{l+m+mi}{0}\PYG{p}{]}\PYG{p}{,} \PYG{p}{[}\PYG{l+m+mi}{0}\PYG{p}{]}\PYG{p}{,} \PYG{n}{color}\PYG{o}{=}\PYG{n}{cmap}\PYG{p}{(}\PYG{l+m+mf}{0.}\PYG{p}{)}\PYG{p}{,} \PYG{n}{lw}\PYG{o}{=}\PYG{l+m+mi}{4}\PYG{p}{)}\PYG{p}{,}
                \PYG{n}{Line2D}\PYG{p}{(}\PYG{p}{[}\PYG{l+m+mi}{0}\PYG{p}{]}\PYG{p}{,} \PYG{p}{[}\PYG{l+m+mi}{0}\PYG{p}{]}\PYG{p}{,} \PYG{n}{color}\PYG{o}{=}\PYG{n}{cmap}\PYG{p}{(}\PYG{o}{.}\PYG{l+m+mi}{5}\PYG{p}{)}\PYG{p}{,} \PYG{n}{lw}\PYG{o}{=}\PYG{l+m+mi}{4}\PYG{p}{)}\PYG{p}{,}
                \PYG{n}{Line2D}\PYG{p}{(}\PYG{p}{[}\PYG{l+m+mi}{0}\PYG{p}{]}\PYG{p}{,} \PYG{p}{[}\PYG{l+m+mi}{0}\PYG{p}{]}\PYG{p}{,} \PYG{n}{color}\PYG{o}{=}\PYG{n}{cmap}\PYG{p}{(}\PYG{l+m+mf}{1.}\PYG{p}{)}\PYG{p}{,} \PYG{n}{lw}\PYG{o}{=}\PYG{l+m+mi}{4}\PYG{p}{)}\PYG{p}{]}

\PYG{n}{fig}\PYG{p}{,} \PYG{n}{ax} \PYG{o}{=} \PYG{n}{plt}\PYG{o}{.}\PYG{n}{subplots}\PYG{p}{(}\PYG{n}{figsize}\PYG{o}{=}\PYG{p}{(}\PYG{l+m+mi}{10}\PYG{p}{,} \PYG{l+m+mi}{5}\PYG{p}{)}\PYG{p}{)}
\PYG{n}{lines} \PYG{o}{=} \PYG{n}{ax}\PYG{o}{.}\PYG{n}{plot}\PYG{p}{(}\PYG{n}{data}\PYG{p}{)}
\PYG{n}{ax}\PYG{o}{.}\PYG{n}{legend}\PYG{p}{(}\PYG{n}{custom\PYGZus{}lines}\PYG{p}{,} \PYG{p}{[}\PYG{l+s+s1}{\PYGZsq{}}\PYG{l+s+s1}{Cold}\PYG{l+s+s1}{\PYGZsq{}}\PYG{p}{,} \PYG{l+s+s1}{\PYGZsq{}}\PYG{l+s+s1}{Medium}\PYG{l+s+s1}{\PYGZsq{}}\PYG{p}{,} \PYG{l+s+s1}{\PYGZsq{}}\PYG{l+s+s1}{Hot}\PYG{l+s+s1}{\PYGZsq{}}\PYG{p}{]}\PYG{p}{)}\PYG{p}{;}
\end{sphinxVerbatim}

\noindent\sphinxincludegraphics{{notebooks_2_0}.png}

\sphinxAtStartPar
There is a lot more that you can do with outputs (such as including interactive outputs)
with your book. For more information about this, see \sphinxhref{https://jupyterbook.org}{the Jupyter Book documentation}


\part{Methodologie, onderzoeksmethode of onderzoeksopzet}


\chapter{Methoden}
\label{\detokenize{methoden:methoden}}\label{\detokenize{methoden::doc}}

\part{Referentie}


\chapter{Referentie}
\label{\detokenize{referentie:referentie}}\label{\detokenize{referentie::doc}}
\begin{sphinxthebibliography}{HdHPK14}
\bibitem[HdHPK14]{markdown:id3}
\sphinxAtStartPar
Christopher Ramsay Holdgraf, Wendy de Heer, Brian N. Pasley, and Robert T. Knight. Evidence for Predictive Coding in Human Auditory Cortex. In \sphinxstyleemphasis{International Conference on Cognitive Neuroscience}. Brisbane, Australia, Australia, 2014. Frontiers in Neuroscience.
\end{sphinxthebibliography}







\renewcommand{\indexname}{Index}
\printindex
\end{document}